\documentclass{article}

\usepackage[final]{style}
\usepackage[utf8]{inputenc} % allow utf-8 input
\usepackage[T1]{fontenc}    % use 8-bit T1 fonts
\usepackage{hyperref}       % hyperlinks
\usepackage{url}            % simple URL typesetting
\usepackage{booktabs}       % professional-quality tables
\usepackage{amsfonts}       % blackboard math symbols
\usepackage{nicefrac}       % compact symbols for 1/2, etc.
\usepackage{microtype}      % microtypography
\usepackage{verbatim}
\usepackage{graphicx}       % for figures

\usepackage{amsmath}

\title{Notes of Computer Vision}

\author{
  Xiangli Chen \\
  Amazon Robotics\\
  % \texttt{\{name\}@cs.stanford.edu} \\
}

\begin{document}

\maketitle
%\tableofcontents

\section{Introduction}
\section{Foundation}
\subsection{Linear Algebra}
\paragraph{Rotation Equation}
2D counter-clockwise rotation by an angle $\theta$
\begin{align*}
x' = \cos \theta x - \sin \theta y \\
y' = \cos \theta y + \sin \theta x.
\end{align*}
Matrix formulation is
\[
\begin{bmatrix}
  x' \\ y'
\end{bmatrix}=
\begin{bmatrix}
  cos\theta & -sin\theta \\
  sin\theta & cos\theta
\end{bmatrix}
\begin{bmatrix}
  x \\ y
\end{bmatrix}.
\]
If you rotate the vector 
$x=\begin{bmatrix}1\\0\\\end{bmatrix}$, by an angle $\theta$, 
its new coordinates are 
$\begin{bmatrix}\cos \theta \\\sin \theta \\\end{bmatrix}$ 
and if you rotate the vector 
$y=\begin{bmatrix}0\\1\\\end{bmatrix}$, 
by an angle $\theta$, its new coordinates are 
$\begin{bmatrix}-\sin \theta \\\cos \theta \\\end{bmatrix}$.
This gives the general formula for the new coordinates $(x', y')$ 
of the point $(x, y)$ after rotation.

The 2D rotation matrix (R) satisfies
\begin{itemize}
\item $R\cdot R^T = I$ 
\item $\det (R) = 1$.
\end{itemize}

% References
\small
\bibliographystyle{plain}
\bibliography{biblio}
\end{document}
